% LaTeX Article Template - customizing header and footer
\documentclass{article}

\newtheorem{thm}{Theorem}

% Set left margin - The default is 1 inch, so the following 
% command sets a 1.25-inch left margin.
\setlength{\oddsidemargin}{0.25in}

% Set width of the text - What is left will be the right margin.
% In this case, right margin is 8.5in - 1.25in - 6in = 1.25in.
\setlength{\textwidth}{6in}

% Set top margin - The default is 1 inch, so the following 
% command sets a 0.75-inch top margin.
\setlength{\topmargin}{-0.25in}

% Set height of the header
\setlength{\headheight}{0.3in}

% Set vertical distance between the header and the text
\setlength{\headsep}{0.2in}

% Set height of the text
\setlength{\textheight}{9in}

% Set vertical distance between the text and the
% bottom of footer
\setlength{\footskip}{0.1in}

% Set the beginning of a LaTeX document
\usepackage{tikz}
\usepackage{multirow}
\usepackage{fullpage}
\usepackage{graphicx}
\usepackage{amsthm}
\usepackage{url}
\usepackage{amssymb}
\usepackage{amssymb}
\usepackage{algpseudocode}
\graphicspath{%
    {converted_graphics/}% inserted by PCTeX
    {/}% inserted by PCTeX
}
%%%%%%%%%%%%%%%%%%%%%%%%%%%%%




\begin{document}\title{Homework $3$\\ Computer Science \\ B551 Spring 2018\\ Hasan Kurban}         % Enter your title between curly braces
\author{Your Name}        % Enter your name between curly braces
\date{\today}          % Enter your date or \today between curly braces
\maketitle

      
% Redefine "plain" pagestyle
\makeatother     % `@' is restored as a "non-letter" character




% Set to use the "plain" pagestyle
\pagestyle{plain}
\section*{Introduction}
The aim of this homework is to get you well-acquainted with $\alpha\beta$. You will turn-in two files \begin{itemize} \item A *pdf with the written answers called \texttt{hw3.pdf} \item  A Python script called gobblet.py\end{itemize}  I am providing this \LaTeX{} document for you to freely use as well. Please enjoy this homework and ask yourself what interests you and then how can you add that interest to it!  Finally, problems 1 and 2 are worth 20 each and problem 3 is worth 120 points. Include that
statement, ``All the work herein is mine."

\newpage
\section*{Homework Questions}
\begin{enumerate}
\item A general search strategy is to work from both the start and goal--think of navigating a maze. Is this 
a sound strategy for a game? Recall that soundness means $\vdash_{Robot}$\ A $\rightarrow\models_{Human}$ A
\item Assume there is a game with three players $A,B,C$. You have an evaluation function h that returns
3 numbers: $h(\sigma) =  \langle A_s, B_s, C_s \rangle$ where $\sigma$ is a state of the game and the numbers reflect the goodness
of the state for the respectively named players. In no more than two paragraphs, is there a way to
modify minimax to work with 3 players? Assume that you cannot exhaustively search the game space
and you're able to generate from a state all the next states.
\item  Implement $\alpha\beta$ for the game of gobblet. Assume that if a state $\sigma$ is repeated, then the game is a draw:

\begin{equation}
\forall(\sigma)\ [\textsf{move}(\textsf{move}(\sigma, \textsf{player1}),\textsf{player2}) = \sigma] \rightarrow (\textsf{game}(\sigma) := \textsf{draw})
\end{equation}

The game should be able to play: human vs. human = h2, human vs. robot = hr (human goes first),
robot vs. human (robot goes first), robot vs. robot = r2. There are two sets of parameters: Level:
beginer = 0, intermediate = 1, expert = 2. This places a limit on the depth of the search. You must
decide experimentally how this is applied; Time: $x$ min. This is the bound on the time you run $\alpha\beta.$ The main function should be called $ \textsf{gobby} (\textsf{players,level,time})$ where players $\in \{\textsf{h2,hr,rh,rr}\}$. The
output should be a sequence of moves with the final state when it's won or drawn.
\\
\\
\textsf{gobby(r2,2,2.5)}
\\
\\
 means robot versus robot, expert, no longer than two minutes and 30 seconds should elapse after the
opponent makes a move. Whatever is unspecified at this point, you must make decisions on.
 
\end{enumerate}
\end{document}
